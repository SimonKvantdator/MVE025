\section{3.13}%
\label{sec:3_13}

Låt $f(z) = \frac{2 z}{z + 2}$. Rita två grafer, en som visar följande sex delmängder $G \subset \complexes$, och en som visar deras bild under $f$.

\subsection{a.}
$G_a = \reals \cup \set{\infty}$.

\paragraph{Lösning:}
Denna linje kommer avbildas på antingen en linje eller en cirkel.
Men det är uppenbart att $f \mid_\reals \from \reals \to \reals$, så $G_a$ avbildas på sig själv.

\subsection{b.}
$G_b = i \reals \cup \set{\infty}$.

\paragraph{Lösning:}
\begin{subequations}
\begin{align}
	0 &\mapsto 0,%
	\label{eq:image_of_0}\\
	%
	\infty &\mapsto 2,%
	\label{eq:image_of_infty}
\end{align}
så $f(G_b)$ är en cirkel som går genom punkterna $0$ och $2$.
Men på grund av symmetri måste $f(G_b)$ vara symmetrisk i realaxeln, så $f(G_b)$ är cirkeln med centrum i $1$ med radie $1$.

\subsection{c.}
$G_c = \set{z \suchthat x = y} \cup \set{\infty}$.

\paragraph{Lösning:}
\begin{align}
	f \from 1 + i &\mapsto \frac{4 + 2 i}{5}.%
	\label{eq:image_of_1+i}
\end{align}
Så $f(G_c)$ är en cirkel som går genom $0$, $2$, och $\frac{4 + 2 i}{5}$.

\subsection{d.}
$G_d = C[0, 2]$.

\paragraph{Lösning:}
\begin{align}
	f \from 2 &\mapsto 1,
	\label{eq:image_of_2}\\
	%
	f \from -2 &\mapsto \infty.%
	\label{eq:image_of_-2}
\end{align}
\end{subequations}
$G_d$ träffar därför $G_a$ i punkten $1$ med vinkeln $\frac{\pi}{2}$.
Eftersom $\infty \in f(G_d)$ är hon en linje.
$G_d$ är alltså linjen $x = 1$.

\subsection{e.}
$G_e = C[1, 1]$.

\paragraph{Lösning:}
\Cref{eq:image_of_0,eq:image_of_2} samt symmetri i realaxeln $\implies$ $f(G_e) = C[\frac{1}{2}, \frac{1}{2}]$.

\subsection{f.}
$G_f = C[-1, 1]$.

\paragraph{Lösning:}
\Cref{eq:image_of_-2} $\implies$ $f(G_f)$ är en linje.
\Cref{eq:image_of_0} $\implies$ $f(G_f)$ korsar realaxeln i $0$(med vinkeln $\frac{\pi}{2}$).
Så $f(G_f) = \set{x = 0}$.


\section{3.14}%
\label{sec:3_14}
Hitta Möbiustransformationer som uppfyller följande.

\subsection{a.}
\begin{align*}
	1 &\mapsto 0,\\
	2 &\mapsto 1,\\
	3 &\mapsto \infty.
\end{align*}

\paragraph{Lösning:}
\begin{align*}
	\frac{z - 1}{z - 3} \cdot \frac{2 - 3}{2 - 1}.
\end{align*}

\subsection{b.}
\begin{align*}
	1 &\mapsto 0,\\
	1 + i &\mapsto 1,\\
	2 &\mapsto \infty.
\end{align*}

\paragraph{Lösning:}
\begin{align*}
	\frac{z - 1}{z - 2} \cdot \frac{(1 + i) - 2}{(1 + i) - 1}.
\end{align*}

\subsection{c.}
\begin{align*}
	0 &\mapsto i,\\
	1 &\mapsto 1,\\
	\infty &\mapsto -i.
\end{align*}

\paragraph{Lösning:}
\begin{align*}
	-i \frac{z - i}{z + i}.
\end{align*}


\section{3.18}%
\label{sec:3_18}
Hitta en Möbiustransformation som avbildar $D[0, 1]$ på $G \definedas \set{x + i y \in \complexes \suchthat x + y > 0}$.

\paragraph{Lösning:}
Om vi kan hitta en Möbiustransformation som avbildar $\partial D[0, 1]$ på $\partial G$ så finns det två alternativ:
\begin{enumerate}
	\item insidan av $D[0, 1]$ avbildas på insidan av $G$,
	\item insidan av $D[0, 1]^\complement$ avbildas på insidan av $G$.
\end{enumerate}
Det räcker att kolla på bilden av en punkt för att avgöra vilken av dessa som gäller.

Möbiustransformation
\begin{align}
	f(z) = (1 - i) \frac{z + i}{z + 1} 
\end{align}
avbildar
\begin{align*}
	-i &\mapsto 0,\\
	-1 &\mapsto \infty,\\
	\infty &\mapsto 1 - i.
\end{align*}
Så $f(\partial D[0, 1])$ är en linje som går jenom $0$ och $1 + i$, alltså $\partial G$.
Det som återstår att kolla är att $f(D[0, 1])$ hamnar på \enquote{rätt sida om randen}.
Men det kan vi övertyga oss om genom att betrakta $f \from 0 \mapsto 1 + i \in G$.

Om det sista steget hade skitit sig hade man bara behövt definera om $f$ med motsatt tecken.


\section{3.19}%
\label{sec:3_19}
Jacobianen av en avbildning $(u, v) \from \reals^2 \to \reals^2$ är determinanten av Jacobimatrisen
\begin{align}
	J =
	\begin{bmatrix}
		\partialderivative{u}{x} & \partialderivative{u}{y}\\
		\partialderivative{v}{x} & \partialderivative{v}{y}
	\end{bmatrix}.
\end{align}
Visa att om $f = y + iv$ är holomorf så är Jacobianen $\abs{f'(z)}^2$.

\paragraph{Lösning:}
\begin{align*}
	\abs{f'(z)}^2
	\overset{(2.2)}&{=} \abs{u_x + i v_x}^2\\
	%
	&= u_x^2 + v_x^2\\
	%
	\overset{(2.3)}&{=} u_x v_y - v_x u_y\\
	%
	&= \abs{J}.
\end{align*}
