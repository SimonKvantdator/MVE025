

\section{1.1}
Låt $z = 1 + 2 i$, $w = 2 - i$ och beräkna följande:

\subsection{b.}
$\conjugate w - z$.

\paragraph{Lösning:}
\begin{align*}
	\conjugate w - z ={} (2 + i) - (1 + 2 i)\\
	={} 1 - i.
\end{align*}

\subsection{c.}
$z^3$.

\paragraph{Lösning:}
\begin{align*}
	z^3 ={}& (1 + 2 i)^3\\
		={}& 1 + 3 \cdot 1^2 \cdot 2 i + 3 \cdot 1 \cdot (2 i)^2 + (2 i)^3\\
		={}& -11 - 2 i.
\end{align*}

\subsection{d.}
$\operatorname{Re} (w^2 + w)$.

\paragraph{Lösning:}
\begin{align*}
	\operatorname{Re} (w^2 + w) ={}& \operatorname{Re} (2^2 - 2 i + i^2 + 2 - i) = 5.
\end{align*}


\section{1.2}
Hitta real- och imaginärdel av följande uttryck:

\subsection{a.}
$\frac{z - a}{z + a}$ för $a \in \reals$.

\paragraph{Lösning:}
Om uttrycket är väldefinerat borde nämnaren vara nollskiljd, och då även nämnarens konjugat.
Med $z = x + y i$ har vi
\begin{align*}
	\frac{z - a}{z + a} ={}& \frac{x + y i - a}{x + y i + a}\\
	={}& \frac{(x - y i + a) (x + y i - a)}{(x - y i + a) (x + y i + a)}\\
	={}& \frac{-a^2 + i 2 a y + x^2 + y^2}{a^2 + 2 a x + x^2 + y^2}\\
	={}& \frac{-a^2 + x^2 + y^2}{a^2 + 2 a x + x^2 + y^2} + \frac{2 a y}{a^2 + 2 a x + x^2 + y^2} i.
\end{align*}
 

\subsection{b.}
$\frac{3 + 5 i}{1 + 7 i}$.
\paragraph{Lösning:}
\begin{align*}
	\frac{3 + 5 i}{1 + 7 i} ={}& \frac{(1 - 7 i) (3 + 5 i)}{1 + 7^2}\\
	={}& \frac{3 - 21 i + 5 i + 35}{49}\\
	={}& \frac{38}{49} + \frac{-16}{49} i
\end{align*}

\subsection{d.}
$\left( \frac{-1 + i \sqrt 3}{2} \right)^3$.
\paragraph{Lösning:}
\begin{align*}
	\left( \frac{-1 + i \sqrt 3}{2} \right)^3 ={}& \left( 1 \angle \frac{2 \pi}{3} \right)^3\\
	={}& 1 \angle 2 \pi\\
	={}& 1.
\end{align*}


\section{1.3}%
\label{sec:1_3}
Hitta beloppet och konjugatet av följande uttryck:

\subsection{b.}
$z = (2 + i) (4 + 3 i)$.

\paragraph{Lösning:}
\begin{align*}
	\abs{z} ={}& \abs{(2 + i)} \cdot \abs{(4 + 3 i)}\\
	={}& \sqrt{3} \cdot 5.
\end{align*}
\begin{align*}
	\conjugate z ={}& \conjugate{(2 + i)}\, \conjugate{(4 + 3 i)}\\
	={}& (2 - i) (4 - 3 i).
\end{align*}

\subsection{d.}
$z = (1 + i)^6$.

\paragraph{Lösning:}
\begin{align*}
	\abs{z} ={}& \abs{(1 + i)}^6\\
	={}& \sqrt{2}^6.
\end{align*}
\begin{align*}
	\conjugate z ={}& (1 - i)^6
\end{align*}


\section{1.4}%
\label{sec:1_4}
Skriv om till polär form:

\subsection{f.}
$z = \abs{3 - 4 i}$.

\paragraph{Lösning:}
\begin{align*}
	z = \abs{3 - 4 i} \angle 0.
\end{align*}

\subsection{h.}
$z = \left(\frac{1 - i}{\sqrt{3}} \right)^4$.

\paragraph{Lösning:}
\begin{align*}
	z ={}& \left(\frac{1}{\sqrt{3}} \sqrt{2} \exp{-\frac{2 \pi}{8} i} \right)^4\\
	={}& \frac{4}{9} \exp{-\pi i}.
\end{align*}


\section{1.8}%
\label{sec:1_8}
Använd $pq$-regeln för att lösa följande ekvationer:

\subsection{a.}
$z^2 + 25 = 0$.

\paragraph{Lösning:}
\begin{align*}
	z = \pm 5.
\end{align*}


\subsection{b.}
$2 z^2 + 2 z + 5 = 0$.

\paragraph{Lösning:}
\begin{align*}
	z ={}& \frac{-1 \pm \left(1 - 4 \frac{5}{2}\right)^\frac{1}{2}}{2}\\
	={}& \frac{-1 \pm \sqrt{4 \frac{5}{2} - 1} i}{2}.
\end{align*}


\section{1.11}%
\label{sec:1_11}
Hitta samtliga lösningar till följande ekvationer:

\subsection{a.}
$z^6 = 1$.

\paragraph{Lösning:}
\begin{align*}
	z^6 ={}& \exp{n 2 \pi i}\\
	z ={}&  \exp{n \frac{\pi}{3} i},
\end{align*}
för $n \in \integers$.

\subsection{c.}
$z^6 = -9$.

\paragraph{Lösning:}
\begin{align*}
	z^6 ={}& 9 \exp{(\pi + n 2 \pi) i}\\
	z ={}& 3^\frac{1}{3} \exp{(\frac{\pi}{6} + n \frac{\pi}{3}) i}.
\end{align*}


\section{1.22}%
\label{sec:1_22}
Bevisa att, för alla $z$ och $w \in \complexes$,
\begin{align}
	\conjugate{z + w} ={}& \conjugate z + \conjugate w,%
	\label{eq:conjugate_distributes_over_addition}\\
	%
	\conjugate{z w} ={}& \conjugate z \conjugate w,%
	\label{eq:conjugate_distributes_over_multiplication}\\
	%
	\conjugate{z / w} ={}& \conjugate z / \conjugate w,%
	\label{eq:conjugate_distributes_over_division}\\
	%
	\conjugate{\conjugate z} ={}& z,%
	\label{eq:conjugate_is_its_own_inverse}\\
	%
	\abs{\conjugate z} ={}& \abs{z},%
	\label{eq:abs_conj=abs}\\
	%
	\abs{z}^2 ={}& z \conjugate z,%
	\label{eq:abs_z2=z_conj_z}\\
	%
	\operatorname{Re} z ={}& \frac{1}{2} (z + \conjugate z),%
	\label{eq:Re_z_in_terms_of_conjugate}\\
	%
	\operatorname{Im} z ={}& \frac{1}{2 i} (z - \conjugate z),%
	\label{eq:Im_z_in_terms_of_conjugate}\\
	%
	\conjugate{\exp{i \phi}} ={}& \exp{-i \phi}%
	\label{eq:conjugation_commutes_with_exp}.
\end{align}

\paragraph{Lösning:}%
\label{par:solution_}
Om vi väljer att definera konjugering som spegling i realaxeln blir många av dom här identiteterna någorlunda uppenbara.

Om addition defineras enligt prallellogramregeln så säger \cref{eq:conjugate_distributes_over_addition} att man får samma parallellogram om man speglar varje ben för sig som om man speglar hela på en gång, vilket man ju får. 

Om multiplikation av två komplexa tal defineras som att man multiplicerar deras längder och adderar deras vinklar är det också uppenbart att jag kan spegla varje tal för sig och få samma svar som om jag speglar efter att ha multiplicerat (längden påverkas inte av spegling och vinklarna blir samma pga $-(\theta + \phi) = -\theta -\phi$).
Vi har då visat \cref{eq:conjugate_distributes_over_multiplication}.
\Cref{eq:conjugate_distributes_over_division} följer på samma sätt.

Spegling i realaxeln är sin egen invers, varför \cref{eq:conjugate_is_its_own_inverse}.

Längder påverkas inte av spegling, varför \cref{eq:abs_conj=abs}.

$z \conjugate z$ har argumentet $\theta - \theta = 0$ och är alltså reellt.
Dess längd är $\abs{z}^2$.
Därav \cref{eq:abs_z2=z_conj_z}.

\cref{eq:Re_z_in_terms_of_conjugate,eq:Im_z_in_terms_of_conjugate} förklaras av \cref{fig:Re_and_Im_z_in_terms_of_conjugate}.
\begin{figure}[ht]
	\centering
	\includegraphics[width=0.7\linewidth]{figures/Re_and_Im_z_in_terms_of_conjugate.png}
	\caption{$z = x + y i$.}%
	\label{fig:Re_and_Im_z_in_terms_of_conjugate}
\end{figure}

\Cref{eq:conjugation_commutes_with_exp} är trivial om man definerar exponentiering av komplexa tal enligt
\begin{align*}
	\exp{x + y i} = \exp{x} \angle y.
\end{align*}


\section{1.23}%
\label{sec:1_23}
Skissa följande delmängder till det komplexa talplanet:

\subsection{b.}
$\set{z \in \complexes \suchthat \abs{z - 1 + i} \leq 2}$.

\paragraph{Lösning:}
Detta är en disk med centrum i $1 - i$ och radie $2$.
Randen är inkluderad.

\subsection{d.}
$\set{z \in \complexes \suchthat \abs{z - i} + \abs{z + i} = 3}$.
\paragraph{Lösning:}
Detta är en ellips med brännpunkterna $i$ och $-i$ och med större radie $\frac{3}{2}$ ($\frac{3}{2} i$ ligger i mängden).

\subsection{f.}
$\set{z \in \complexes \suchthat \abs{z - 1} = 2 \abs{z + 1}}$.
\paragraph{Lösning:}
Mängden, kalla den $A$, är preimage av en cirkel med radie $2$ under avbildningen $f \from z \mapsto \frac{z - 1}{z + 1}$.
Vi har att
\begin{align*}
	f^{-1}(z) ={}& \frac{1 + z}{1 - z}\\
	={}& \frac{2}{1 - z} - 1
\end{align*}
$f$ och $f^{-1}$ tillhör en speciell mängd av funktioner som kallas \emph{Möbiustransformationer}.
Möbiustransformationer karaktäriseras av att de avbildar cirklar på cirklar.
Man kan se att $f^{-1}$ bevarar cirklar genom att övertyga sig själv om att varje komposant i $f = (z \mapsto z - 1) \circ (z \mapsto 2 z) \circ (z \mapsto \frac{1}{z}) \circ (z \mapsto -z + 1)$ bevarar cirklar.
En cirkel är entydigt bestämd av tre punkter, så eftersom $-\frac{1}{3}$ och $-3 \in A$ och eftersom $A$ är symmetrisk i realaxeln har vi att $A$ är cirkeln centrerad i $-\frac{5}{3}$ med radie $\frac{4}{3}$.

\subsection{g.}
$\set{z \in \complexes \suchthat \operatorname{Re} z^2 = 1}$.

\paragraph{Lösning:}
Mänged är preimage av linjen $\Re z = 1$ under avbildningen $z \mapsto z^2$.
Så vi har att preimage av $z = 1 + y i$ är
\begin{align*}
	\pm {(1 + y i)}^\frac{1}{2}.
\end{align*}










\section{1.24}%
\label{sec:1_24}
Låt $p$ vara ett polynom med reella koefficienter.
Visa att

\subsection{a.}
$\conjugate{p(z)} = p(\conjugate z)$.
\paragraph{Lösning:}
Vi gör ett induktonsbevis.
Basfallet är en konstant reell funtktion, så påståendet stämmer där \checkmark.

Om vi betecknar induktionshypotesen med $H$ har vi att
\begin{align*}
	\conjugate{p_{n + 1}(z)} ={}&
	\conjugate{a_{n + 1} z^{n + 1} + a_n z^n + \dots + a_0}\\
	%
	\overset{\text{\cref{eq:conjugate_distributes_over_addition}}}{=}{}&
	\conjugate{a_{n + 1} z^{n + 1}} + \conjugate{a_n z^n + \dots + a_0}\\
	%
	\overset{\text{\cref{eq:conjugate_distributes_over_multiplication}}}{=}{}&
	a_{n + 1} \conjugate{z}^{n + 1} + \conjugate{a_n z^n + \dots + a_0}\\
	%
	\overset{H}{=}{}&
	a_{n + 1} \conjugate{z}^{n + 1} + a_n \conjugate{z}^n + \dots + a_0\\
	%
	={}& p_{n + 1}(\conjugate z).
\end{align*}


\subsection{b.}
$p(z) = 0$ omm $p(\conjugate z) = 0$.
\paragraph{Lösning:}
Följer av \textbf{a.} och att $\conjugate 0 = 0$.


\section{1.25}%
\label{sec:1_25}
Bevisa den omvända triangelolikheten
\begin{align}\label{eq:inverse_triangle_inequality}
	\abs{z_1 - z_2} \geq \abs{\abs{z_1} - \abs{z_2}}
\end{align}
\paragraph{Lösning:}
Betrakta \cref{fig:inverse_triangle_inequality}.
VL i \cref{eq:inverse_triangle_inequality} har ett minimum när $\theta_{12} = 0$.
Då är VL och HL lika.
HL beror inte på $\theta_{12}$.
Alltså följer påståendet.

\begin{figure}[ht]
	\centering
	\includegraphics[width=0.4\linewidth]{figures/inverse_triangle_inequality.png}
	\caption{$\theta_{12}$ är vinkeln mellan $z_1$ och $z_2$.}%
	\label{fig:inverse_triangle_inequality}
\end{figure}


\section{1.26}%
\label{sec:1_26}
Använd inversa triangelolikheten för att visa att
\begin{align}\label{eq:1_26}
	\abs{\frac{1}{z^2 - 1}} \leq \frac{1}{3}
\end{align}
för alla $z \in C[0, 2]$.

\paragraph{Lösning:}
Eftersom VL och HL är positiva är \cref{eq:1_26} ekvivalent med
\begin{align}\label{eq:1_26_modified}
	\abs{z^2 - 1} \geq 3.
\end{align}
Men om $\abs{z} = 2$ är \cref{eq:1_26_modified} triangelolikheten för $z_1 = z^2$, $z_2 = 1$.


\section{1.27}%
\label{sec:1_27}
Skissa följande delmängder till $\complexes$ och avgör om dom är öppna, slutna, begränsade, sammanhängande.

\subsection{a.}
$\set{z \in \complexes \suchthat \abs{z + 3} < 2}$.
\paragraph{Lösning:}
Öppen disk med radie $2$ och centrum i det komplexa talet $3$.


\subsection{b.}
$\abs{\operatorname{z}} < 1$.
\paragraph{Lösning:}
Ett band mellan $y = \pm 1$.
Öppet, sammanhängande, ej begränsat.


\subsection{c.}
$0 < \abs{z - 1} < 2$.
\paragraph{Lösning:}
Öppen punkterad disk med radie $2$ och centrum i det komplexa talet $1$.
Öppen, sammanhängande, begränsad.

\subsection{d.}
$\abs{z - 1} + \abs{z + 1} = 2$.
\paragraph{Lösning:}
Ett linjesegment från $1$ till $-1$.
Öppet, sammanhängande, begränsat.


\section{1.29}%
\label{sec:1_29}
Låt $G = [-2, -1] \cup D(0, 1) \cup \set{1, 2}$.

\subsection{b.}
Vilka punkter är $G$:s inre punkter?
\paragraph{Lösning:}
$D(0, 1)$.

\subsection{c.}
Vilka punkter är $G$:s randpunkter?
\paragraph{Lösning:}
\begin{align*}
	\partial G \definedas \overline{G} \cap \overline{G^\complement} = [-2, -1] \cup C(0, 1) \cup \set{1, 2}.
\end{align*}
Kom ihåg att $z \in \partial G$ inte nödvändigtvis behöver betyda att $z \in G$.

\subsection{d.}
Vilka punkter är $G$:s isolerade punkter?
\paragraph{Lösning:}
$z$ isolerad innebär att det finns $\epsilon$ s.a.\ $D(z) \cap G = \set{z}$.
Endast en punkt uppfyller detta, $2$.
Kom ihåg att $z$ isolerad punkt till $G$ nödvändigtvis betyder att $z \in G$.


\section{1.33}%
\label{sec:1_33}
Parametrisera följande kurvor:

\subsection{a.}
$C[1 + i, 1]$, orienterad moturs.
\paragraph{Lösning:}
$z(t) = \exp{i t} + 1 + i$, $t \in [0, 2 \pi]$.

\subsection{b.}
Linjesegmentet från $z_1 = -1 - i$ till $z_2 = 2 i$.
\paragraph{Lösning:}
$z(t) = (1 - t) z_1 + t z_2$, $t \in [0, 1]$.

\subsection{e.}
Ellipsen centrerad i $0$ med större radie $2$ och mindre radie $\sqrt 3$, orienterad moturs.
\paragraph{Lösning:}
Låt $f(t) = \exp{i t}$ och $g(x + i y) = 2 x + \sqrt 3 i y$.
Då är $g \circ f$ en parametrisering.



