%! TEX root = /home/simon/Documents/komplexanalysrövningsledning/lösningsförslag/main.tex

\section{7.25}%
\label{sec:7_25}
Hitta potensserier och bestäm deras konvergensradier för följande funktioner.

\section{a)}
\begin{align*}
	\frac{1}{1 + 4 z}.
\end{align*}

\paragraph{Lösning:}
Uppgiften specificerar ingen punkt att centrera potensserierna i, men jag väljer att utveckla kring $0$.

Vi vill använda oss av den geometriska serien
\begin{align}\label{eq:geometric_series}
	\frac{1}{1 - w} = \sum_{k \geq 0} w^k.
\end{align}
Med $w = -4 z$ blir \cref{eq:geometric_series} 
\begin{align*}
	\frac{1}{1 + 4 z} ={}& \sum_{k \geq 0} (- 4 z)^k.
\end{align*}
Eftersom \cref{eq:geometric_series}:s konvergensradie är $1$ är denna series konvergensradie $\frac{1}{4}$.

\section{b)}
\begin{align*}
	\frac{1}{3 - \frac{z}{3}}.
\end{align*}

\paragraph{Lösning:}
I \cref{eq:geometric_series}, välj $w = \frac{z}{6}$.
Då har vi att
\begin{align*}
	\frac{1}{3 - \frac{z}{2}} ={}& \frac{1}{3} \sum_{k \geq 0} (\frac{z}{6})^k
\end{align*}
med konvergensradie $6$

\section{c)}
\begin{align*}
	\frac{z^2}{(4 - z)^2}.
\end{align*}

\paragraph{Lösning:}
Om vi deriverar den geometriska \cref{eq:geometric_series} serien får vi
\begin{align}
	\frac{1}{(1 - w)^2} ={}& \sum_{k \geq 1} k w^{k - 1}\nonumber\\
	={}& \sum_{k \geq 0} (k + 1) w^k.%
	\label{eq:derivative_of_geometric_series}
\end{align}
Om vi tänker på fortsan så kommer vi ihåg att om en funktionsföljd är likformigt konvergent så kan vi derivera och gå i gräns i vilken ordning vi vill.
Eftersom potensserier konvergerar likformigt på alla slutna delmängder till deras öppna konvergensdisk så gäller \cref{eq:derivative_of_geometric_series} på hela \cref{eq:geometric_series}:s öppna konvergensdisk.
\cref{eq:derivative_of_geometric_series}:s konvergensradie är alltså minst lika stor som \cref{eq:geometric_series}:s.
Om man stoppar in $w = 1$ är det även uppenbart att konvergensradien inte blir större.

Välj nu $w = \frac{z}{4}$.
Vi erhåller
\begin{align*}
	\frac{z^2}{(4 - z)^2} ={}& \frac{1}{16} z^2 \sum_{k \geq 0} (k + 1) (\frac{z}{4})^k\\
	={}& \frac{1}{16} \sum_{k \geq } (k + 1) (\frac{z}{4})^{k + 2}
\end{align*}
med konvergensradie $4$.


\section{7.27}%
\label{sec:7_27}

\subsection{a)}
Antag att talföljden $c_k$ är begränsad av något tal $M$.
Visa att konvergensradien hos $\sum_{k \geq 0} c_k z^k$ inte är mindre än $1$.

\paragraph{Lösning:}
Om $\abs{z} < 1$ har vi att
\begin{align*}
	\abs{\sum_k c_k z^k} \overset{\text{$\triangle$-inequality}}&{\leq} \sum_k \abs{c_k z^k}\\
	&= \sum_k \abs{c_k} \abs{z}^k\\
	&\leq \sum_k M \abs{z}^k\\
	&= M \sum_k \abs{z}^k\\
	&\leq \infty.
\end{align*}

\subsection{b)}
Antag att talföljden $c_k$ inte konvergerar mot 0.
Visa att konvergensradien hos $\sum_{k \geq 0} c_k z^k$ inte är större än $1$.

\paragraph{Lösning:}
Om $c_k$ inte konvergerar mot $0$ så finns det ett tal $\epsilon > 0$ s.a.\ $\abs{c_k} \geq \epsilon$ för oändligt många $k$.
Vi har då att funktionsföljden $f_n(z) = \sum_{k = 0}^n c_k z^k$ omöjligt kan konvergera om $\abs{z} > 1$ eftersom för varje $N \in \naturals$ kommer jag kunna hitta ett $n > N$ s.a.\ (beloppet av) skillnaden mellan $f_n(z)$ och $f_{n + 1}(z)$ är större än $\epsilon \abs{z}^n > \epsilon$.


\section{7.28}%
\label{sec:7_28}

Hitta potensserier centrerade i $1$ och deras konvergensradie för följande funktioner:

\subsection{a)}
\begin{align*}
	\frac{1}{z}.
\end{align*}

\paragraph{Lösning:}
\begin{align*}
	\frac{1}{z} ={}& \frac{1}{1 - (1 - z)},
\end{align*}
så välj $w = 1 - z$ i \cref{eq:geometric_series}.
Vi erhåller
\begin{align*}
	\frac{1}{z} ={}& \sum_{k \geq 0} (1 - z)^k\\
	={}& \sum_{k \geq 0} (-1)^k (z - 1)^k
\end{align*}
med konvergensradie $1$.


\subsection{b)}
\begin{align*}
	\operatorname{Log} z.
\end{align*}

\paragraph{Lösning:}
Vi integrerar \textbf{a)} och använder att $\operatorname{Log} 1 = 0$:
\begin{align*}
	\operatorname{Log} z = -\sum_{k \geq 0} \frac{(-1)^k}{k} (z - 1)^{k + 1}.
\end{align*}
Samma argument med likformig konvergens som i 7.25 \textbf{c)} ger oss konvergensradien $1$. 




