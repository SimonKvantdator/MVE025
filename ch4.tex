\section{4.34}%
\label{sec:4.34}

Beräkna
\begin{align*}
	I(r) = \int_{C[-2i, r]} \frac{\dd{z}}{z^2 + 1}
\end{align*}
För alla $r \neq 1$, $3$.

\paragraph{Lösning:}
Vi börjar med att skriva ner två satser vi kommer använda oss av.
\begin{theorem*}[Cauchys sats]
	Om $f$ är holo i ett område $G$ och $\gamma$ och $\lambda$ är två styckvis släta kurvor i $G$ s.a.\ $\gamma \sim_G \lambda$, då är
	\begin{align}
		\int_\gamma f = \int_\lambda f.
	\end{align}
\end{theorem*}
\begin{theorem*}[Cauchys integralformel]
	Om $f$ är holo i ett område $G$ och $\gamma$ är en positivt orienterad, enkel, sluten, styckvis slät kurva s.a.\ $w \in \operatorname{int} \gamma$ och $\gamma \sim_G 0$, då är
	\begin{align}\label{eq:cauchy_integral_formula}
		f(w) = \frac{1}{2 \pi i} \int \frac{f(z)}{z - w} \dd{z}.
	\end{align}
\end{theorem*}

Idén är att vi, för olika $r$, kommer kunna skriva om integranden $g(z) = \frac{1}{z^2 + 1}$ på formen \cref{eq:cauchy_integral_formula} på lite olika sätt.
Vi har tre fall
\begin{enumerate}
	\item $\gamma_1 = C[-2i, r]$, $r < 1$,
	\item $\gamma_2 = C[-2i, r]$, $1 < r < 3$,
	\item $\gamma_3 = C[-2i, r]$, $r < 3$.
\end{enumerate}
\Cref{fig:some_gammas} visar ett par exempel på hur dessa tre olika fall kan se ut.

\begin{figure}[ht]
	\centering
	\includegraphics[width=0.5\textwidth]{figures/some_gammas.png}
	\caption{}
	\label{fig:some_gammas}
\end{figure}

I fall 1 ser vi att $g$ inte har nåra singulariter i $\operatorname{int} \gamma_1$.
Därför är $I = 0$ (detta är en uppenbar följdsats till Cauchys sats, men även tillCauchys integralformel genom att välja $(z + 2 i) g(z)$).

I fall 2 ser vi att $g$ har en singularitet i $z = -i$.
Vi kan faktorisera $g(z) = \frac{1}{(z + i) (z - i)}$, så definera $g_2$ enligt $g(z) = \frac{g_2(z)}{z + i}$.
Idén är nu att $g_2(z) = \frac{1}{z - i}$ är holo på $\operatorname{int} \gamma_2$, så allt är upplagt för att använda Cauchys integralformel.
Det är alltså bara att stoppa in $f = g_2$ och $w = -i$.
Vi erhåller
\begin{align*}
	I ={}& \int_{\gamma_2} \frac{g_2(z)}{z + i} \dd{z}\\
	={}& 2 \pi i g_2(-i)\\
	={}& -\pi.
\end{align*}

I fall 3 är saker lite mer kompelicerade, $g$ har två singulariteter i $\operatorname{int} \gamma_3$.
Vi hanterar detta genom att deformera $\gamma_3$ till $\gamma_3'$ enligt \cref{fig:gamma3'}.
\begin{figure}[ht]
	\centering
	\includegraphics[width=0.5\textwidth]{figures/gamma3'.png}
	\caption{Cauchys sats säger att vi har $\int_{\gamma_3} f = \int_{\gamma_3'} f$.}
	\label{fig:gamma3'}
\end{figure}

$\gamma_3'$ består av två delar: en cirkel med centrum i $i$ och en cirkel med centrum i $-i$.
I var och en av dessa två cirklar kan vi applicera Cauchys integralformel!
Vi får att
\begin{align*}
	\int_{\gamma_3} f ={}& \int_{\gamma_3'} f\\
	={}& \int_{C[i, \rho]} f + \int{C[-i, \rho]} f\\
	&\{ g_3(z) \definedas 1/(z + i) \}\\
	={}& \int_{C[i, \rho]} \frac{g_3(z)}{z - i} + \int_{C[-i, \rho]} \frac{g_2(z)}{z + i}\\
	={}& \pi - \pi\\
	={}& 0.
\end{align*}

