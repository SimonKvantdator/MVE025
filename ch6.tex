%! TEX root = /home/simon/Documents/komplexanalysrövningsledning/lösningsförslag/main.tex

\section{6.11}%
\label{sec:6_11}
Visa att, om $U \from \reals^2 \to \reals$ är harmonisk och begränsad på $\complexes$, så är $U$ konstant.

\paragraph{Lösning:}
Vi statear vi statear två användbara satser:
\begin{theorem*}[6.6]
	Låt $u \from \reals^2 \to \reals$ vara harmonisk på ett enkelt sammanhängande område.
	Då finns det en funktion $v \from \reals^2 \to \reals$ s.a.\ $f = u + i v$ är holomorf på $G$.
\end{theorem*}
\begin{theorem*}[Liouvilles sats]
	Alla hela begränsade funktioner är konstanta.
\end{theorem*}

Definera $F$ enligt sats 6.6 s.a.\ $F = U + i V$ är hel ($\complexes$ är ett enkelt sammanhängande område).
Betrakta nu den sammansatta funktionen $G = \operatorname{exp} \circ F$.
$G$ är en sammansättning av två hela funktioner och är därför också hel.
Men $G$ är också begränsad eftersom $F$:s realdel är begränsad.
Enligt Liouvilles sats är då $G$ konstant och därför är även $F$ konstant.
