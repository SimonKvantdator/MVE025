%! TEX root = /home/simon/Documents/komplexanalysrövningsledning/lösningsförslag/main.tex

\section{Laplace.1}%
\label{sec:laplace_1}
Beräkna Laplacetransformen av följande funktioner:

\subsection{a)}
\begin{align*}
	f(t) = \sinh{A t}.
\end{align*}

\paragraph{Lösning:}
\begin{align*}
	\laplacetransform{f}(s) ={}& \int_0^\infty f(t) \exp{-s t} \dd{t}\\
	={}& \int_0^\infty \sinh{A t} \exp{-s t} \dd{t}\\
	={}& \int_0^\infty \left( \frac{1}{2} \exp{A t} - \frac{1}{2} \exp{-A t} \right) \exp{-s t} \dd{t}\\
	={}& \int_0^\infty \left( \frac{1}{2} \exp{[A - s] t} - \frac{1}{2} \exp{[-A - s] t} \right) \exp{-s t} \dd{t}\\
	={}& \left[ \frac{1}{2(A - s)} \exp{[A - s] t} - \frac{1}{2(-A - s)} \exp{[-A - s] t} \right]_0^\infty\\
	={}& -\frac{1}{2(A - s)} + \frac{1}{2(-A - s)}\\
	={}& \frac{A}{(s - A) (s + A)}.
\end{align*}
Kom ihåg att det dyker upp minustecken när vi partialintegrerar och när vi evaluerar i den vänstra gränsen.

\subsection{c)}
\begin{align}
	f(t) = t^2 \exp{a t}.
\end{align}

\paragraph{Lösning:}
\begin{align}
	\laplacetransform{f}(s) ={}& \int_0^\infty f(t) \exp{-s t} \dd{t}\\
	={}& \int_{0}^{\infty} t^2 \exp{[a - s] t} \dd{t}\\
	={}& \int_{0}^{\infty} -\frac{2 t}{a - s} \exp{[a - s] t} \dd{t}\\
	={}& \int_{0}^{\infty} \frac{2}{(a - s)^2} \exp{[a - s] t} \dd{t}\\
	={}& \left[ \frac{2}{(a - s)^3} \exp{[a - s] t} \right]_0^\infty\\
	={}& -\frac{2}{(a - s)^3}.
\end{align}


\section{Laplace.3}%
\label{sec:laplace_3}
Lös följande begynnelsevärdesproblem:

\subsection{b)}
\begin{align}\label{eq:differential_equation_that_i_want_to_laplace_transform}
	u'' + u = \theta(t - \pi)
\end{align}
där $u(0) = u'(0) = 0$ och där $\theta$ är Heavisides stegfunktion.

\paragraph{Lösning:}
Vi kommer bland anna använda följande
\begin{itemize}
	\item $\laplacetransform{u'}(s) = s \laplacetransform{u}(s) - u(0)$,
	\item $\laplacetransform{u''}(s) = s^2 \laplacetransform{u}(s) - s u'(0) - u(0)$,
	\item $g(t) = f(t - t_0) \implies \laplacetransform{g}(s) = \exp{-t_0 s} \laplacetransform{f}(s)$.
\end{itemize}
Eftersom Laplacetransformen antar att alla funktioner är $0$ för $t < 0$ har vi att $\laplacetransform{\theta}(s) = \laplacetransform{1}(s) = \frac{1}{s}$.
Laplacetransformen av \cref{eq:differential_equation_that_i_want_to_laplace_transform} är då
\begin{align}
	s^2 \laplacetransform{u} + \laplacetransform{u} = \frac{\exp{-\pi s}}{s}.
\end{align}
Här har vi an
Av vilket följer att
\begin{align*}
	\laplacetransform{u}(s) = \frac{\exp{-\pi s}}{s (s^2 + 1)} = \exp{-\pi s} \left(\frac{1}{s} - \frac{s}{s^2 + 1}\right).
\end{align*}
Om vi identifierar $\Laplacetransform{\cos}(s) = \frac{s}{1 + s^2}$ kan vi göra en invers Laplacetransform och komma fram till att
\begin{align*}
	u(t) = \theta(t - \pi) \left(1 - \cos{t} \right).
\end{align*}
Det står att svaret ska bli $\theta(t - \pi) (1 + \cos{t})$ men jag vet inte riktigt var jag skulle få plustecknet från?..


\section{Laplace.4}%
\label{sec:laplace_4}
Hitta funktioner med följande Laplacetransformer.

\subsection{a)}
\begin{align}
	\laplacetransform{f}(s) = \frac{s}{(s^2 + 1)^2}.
\end{align}

\paragraph{Lösning:}
Enligt sats 3.2 är laplacetransformen bijektiv, så vi vet att det måste gå att hitta en funktion med denna Laplacetransform.
Även om sats 3.2 ger oss ett uttryck på sluten form är det lite bökigt att räkna med, så vi satsar på att gå baklänges.
Typ som att vi integrerar.
Med vetskapen om att det alltså är möjligt att hitta $f$ gör vi vår första observation:
\begin{align*}
	\derivative{}{s} \left( -\frac{1}{2 (s^2 + 1)} \right) = \frac{s}{(s^2 + 1)^2}.
\end{align*}
Det som står innanför derivatan är bilden av en Laplacetransform:
\begin{align*}
	\Laplacetransform{\sin}(s) = \frac{1}{1 + s^2}.
\end{align*}
Med hjälp av proposition 3.3 \textbf{f)},
\begin{align*}
	\Laplacetransform{t f(t)}(s) = -\derivative{}{s} \laplacetransform{f}(s),
\end{align*}
har vi då att
\begin{align*}
	\laplacetransform{f}(s) ={}& \frac{1}{2} \Laplacetransform{t \sin{t}}\\
	f(t) ={}& \frac{1}{2} t \sin{t}.
\end{align*}


\subsection{b)}
\begin{align*}
	\laplacetransform{f}(s) = \frac{1}{(s - a)^4}.
\end{align*}

\paragraph{Lösning:}
Proposition 3.3 \textbf{f)} kan enkelt generaliseras som
\begin{align*}
	\Laplacetransform{t^n f(f)}(s) = (-1)^n \derivative[n]{}{s} \laplacetransform{f}(s).
\end{align*}
Detta kan vi sedan applicera på
\begin{align*}
	\frac{1}{(s - a)^4} ={}& \frac{(-1)^3}{3 \cdot 2} \derivative[3]{}{s} \frac{1}{s - a}\\
	={}& \frac{(-1)^3}{6} \derivative[3]{}{s} \Laplacetransform{\exp{a t}}(s)\\
	={}& \frac{1}{6} \Laplacetransform{t^3 \exp{a t}}(s).
\end{align*}
Alltså
\begin{align*}
	f(t) = \frac{1}{6} t^3 \exp{a t}.
\end{align*}


\section{Z.1}%
\label{sec:z_1}
Beräkna $Z$-transformen av följande taljöljder:

\subsection{a)}
\begin{align*}
	a_k = \frac{2^k}{k!}.
\end{align*}

\paragraph{Lösning:}
Vi vet ju att
\begin{align*}
	\exp{z} = \sum_{k \geq 0} \frac{1}{k!} z^k,
\end{align*}
så
\begin{align*}
	\exp{2 / z} ={}& \sum_{k \geq 0} \frac{2^k}{k!} \left(\frac{1}{z}\right)^k\\
	={}& Z((a_k)).
\end{align*}

\subsection{c)}
\begin{align*}
	a_k = k(k - 1).
\end{align*}

\paragraph{Lösning:}
Vi vet ju att
\begin{align*}
	\frac{1}{1 - z} = \sum_{k \geq 0} z^k.
\end{align*}
Deriverar vi detta två gånger får vi
\begin{align*}
	\frac{2}{(1 - z)^3} ={}& \sum_{k \geq 0} k (k - 1) z^{k - 2}\\
	={}& \frac{1}{z^2}  \sum_{k \geq 0} k (k - 1) z^k.
\end{align*}
Vi får att
\begin{align*}
	\frac{2}{(1 - 1/z)^3} ={}& z^2 \sum_{k \geq 0} k (k - 1) \left(\frac{1}{z}\right)^k\\
	={}& z^2 Z((a_k)).
\end{align*}
Alltså är 
\begin{align*}
	Z((a_k)) ={}& \frac{2 / z^2}{(1 - 1/z)^3}\\
	={}& \frac{2 z}{(z - 1)^3}.
\end{align*}


\section{Z.3}%
\label{sec:z_3}

Använd $Z$-transformen för att lösa
\begin{align}\label{eq:sequence_to_be_Z-transformed}
	a_{k + 1} - a_k = k, \quad a_0 = 0.
\end{align}

\paragraph{Lösning}
Definera \enquote{$S((a_k)) = (a_{k + 1})$}.
Vi har då att
\begin{align*}
	Z\left( S \left((a_k)\right) \right) = z \left(Z((a_k)) - a_0 \right).
\end{align*}
Om vi applicerar detta på \cref{eq:sequence_to_be_Z-transformed} får vi
\begin{align*}
	z \left(Z((a_k)) - a_0\right) - Z((a_k)) = \frac{z}{(z - 1)^2}.
\end{align*}
Efter lite ommöblering:
\begin{align*}
	Z((a_k)) ={}& \frac{1}{z - 1} \left[\frac{z}{(z - 1)^2} - z a_0\right]\\
	={}& \frac{z}{(z - 1)^3}\\
	={}& \frac{z^{-2}}{(1 - \frac{1}{z})^3}\\
	={}& z^{-2} \derivative[2]{}{(\frac{1}{z})} \left[ \frac{1}{2 (1 - \frac{1}{z})} \right]\\
	={}& z^{-2} \sum_{k \geq 0} \frac{k (k - 1)}{2} \left( \frac{1}{z} \right)^{k - 2}\\
	={}& \sum_{k \geq 0} \frac{k (k - 1)}{2} \left( \frac{1}{z} \right)^k.
\end{align*}
Alltså är $a_k = \frac{k (k - 1)}{2}$

\todo[inline]{Ta upp avsnitt 4.1.1 i Nukelawe's Markov chain analysis of overkill and natural regeneration in OSRS Combat som ett bra exempel på vad detta kan användas till.
Nukelawe får en differensekvation för väntevärdet av fightens längd $L$ som han löser m.h.a.\ ansätta talföljden som koefficienter till en holomorf funktion.}

