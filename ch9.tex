%! TEX root = /home/simon/Documents/komplexanalysrövningsledning/lösningsförslag/main.tex
% Kap 9: 1, 2, 5abcd, 6, 7de, 8cd, 9, 11, 14, 15, 17,18, 21abc.


\section{9.1}%
\label{sec:9_1}
Antag att $f$ har ett nollställe av ordning $m$ i $a$.
Förklara varför $\frac{1}{f}$ har en pol av ordning $m$ i a.

\paragraph{Lösning:}
Vi börjar med att skriva ner två satser
\begin{theorem*}[klassifikation av nollställen]
	\begin{enumerate}
		Antag att $G \subset \complexes$ är ett område.
		Antag $f \from G \to \complexes$ holo och har ett nollställe i $a \in G$.
		Då gäller antingen
		\item $f = 0$.
		\item $\exists m \in \integers_{\geq 1}$ och en holomorf funktion $g \from G \to \complexes$ s.a.\ $g(a) \neq 0$ och 
		\begin{align*}
			f(z) = (z - a)^m g(z).
		\end{align*}
	\end{enumerate}
\end{theorem*}

\begin{theorem*}[Följdsats 9.6]
	Antag $f$ holo i en punkterad disk $D^\times[a, R]$.
	Då har $f$ en pol i $a$ omm $\exists m \in \integers_{\geq 1}$ och holo $g \from D^\times[a, R]$ s.a.\ $g(a) \neq 0$ och
	\begin{align*}
		f(z) = (z - a)^{-m} g(z).
	\end{align*}
\end{theorem*}

Vi har alltså att $f(z) = (z - a)^m g(z)$ för något $m$ och något holo $g$ s.a.\ $g(z) \neq 0$.
Men då är $\frac{1}{g}$ holo i ett område kring $a$ och alltså har $\frac{1}{(z - a)^m g(z)}$ en pol av ordning $m$ enligt följdsats 9.6.


\section{9.2}%
\label{sec:9_2}
Hitta poler/hävbar singulariteter hos följande funktioner och bestäm deras ordning.

\subsection{a)}
\begin{align*}
	f(z) = (z^2 + 1)^{-3} (z - 1)^{-4}.
\end{align*}

\paragraph{Lösning:}
Det är enkelt att göra detta om vi lyckas faktorisera $f$.
Vi har att
\begin{align*}
	f = (z + i)^{-3} (z - i)^{-3} (z - 1)^{-4}.
\end{align*}
Så $f$ har
\begin{itemize}
	\item en pol av ordning $3$ i $-i$,
	\item en pol av ordning $3$ i $i$,
	\item en pol av ordning $4$ i $1$.
\end{itemize}

\subsection{b)}
\begin{align*}
	f(z) = z \cot{z}.
\end{align*}

\paragraph{Lösning:}
\begin{remark}\label{remark:ang_produkter_av_mero_och_holo_funktioner}
	Okej, så om jag har en holo funktion $g$ med en pol i $a$ och en holo funktion $h$ s.a.\ $h(a) \neq 0$ så har $g$ och $g h$ en pol av samma ordning i $a$.
	Detta är uppenbart från följdsats 9.6.
\end{remark}

$\cot{z}$ har en pol av ordning $1$ i $0$ och den är $\pi$-periodisk, så den har poler av ordning $1$ i alla punkter på formen $n \pi$, $n \in \integers$.
Eftersom $z$ är nollskiljd överallt förutom i $0$ har vi att $f$ har poler av ordning $1$ i alla punkter på formen $n \pi$ förutom $0$, där har hon en hävbar singularitet.

\subsection{c)}
\begin{align*}
	f(z) = z^{-5} \sin{z}.
\end{align*}

\paragraph{Lösning:}
Taylorutveckling av $\sin$ ger oss att $f$ har en pol av ordning $4$ i $0$.

\subsection{d)}
\begin{align*}
	f(z) = \frac{1}{1 - \exp{z}}.
\end{align*}

\paragraph{Lösning:}
$f$ har en pol i $0$ av ordning $1$ och eftersom $\mathrm{exp}$ är $2 \pi i$-periodisk har vi att $f$ har poler av ordning $1$ i all punkter på formen $n 2 \pi i$, $n \in \integers$.

\subsection{e)}
\begin{align*}
	f(z) = \frac{z}{1 - \exp{z}} 
\end{align*}

\paragraph{Lösning:}

Vi kombinerar \textbf{d)} med \cref{remark:ang_produkter_av_mero_och_holo_funktioner} och får att $f$ har poler i all punkter på formen $n 2 \pi i$ förutom i $0$.
Där är singulariteten hävbar.

\section{9.5}%
\label{sec:9_5}
Evaluera följande integraler när $\gamma = C[0, 3]$.\\
\todo[inline]{Gör 9.6 först.}

\subsection{a)}
\begin{align*}
	\int_\gamma \cot{z} \dd{z}.
\end{align*}

\paragraph{Lösning:}
Vi börjar med att statea residysatsen
\begin{definition*}
	Antag $a$ isolerad singularitet till en holo funktion $f$ med Laurentutveckling $\sum_{k \in \integers} c_k (z - a)^k$.
	Då är $c_{-1}$ $f$:s \emph{residy} i $a$.
	Vi skriver $\residue_{z = a} [f(z)]$.
\end{definition*}
\begin{theorem*}[Residysatsen]
	Antag $f$ mero i ett område $G$ och antag $\gamma$ positivt orienterad, enkel, sluten, styckvis slät kurva, nollhomotop i $G$ som undviker $f$:s singulariteter.
	Då är
	\begin{align}
		\int_\gamma f \dd{z} = 2 \pi i \sum_a \residue_{z = a}[f(z)]
	\end{align}
	där summan tas över $f$:s singulariteter i $\operatorname{int} \gamma$ (garanterat ändligt många).
\end{theorem*}

$\cot{z}$ har en pol i $\operatorname{int} \gamma$
Där är residyn $1$.
Tänk $\cot{z} = 1 / \tan{z} = (1 / z) (z / \tan{z})$ där $z / \tan{z}$ är holo kring $z = 0$.
Därför blir integralen $2 \pi i$.

\subsection{b)}
\begin{align*}
	\int_\gamma z^3 \cos{\frac{3}{z}} \dd{z}.
\end{align*}

\paragraph{Lösning:}
Integranded har en väsentlig singularitet i $0$, så dess Laurentutveckling kommer ha oändligt många termer med negativ potens.
Men det är fine, vi behöver bara veta koefficienten för $\frac{1}{z}$ i dess Laurentutveckling.
Vi Taylorutvecklar ${\cos}$:
\begin{align*}
	z^3 \cos{\frac{3}{z}} = \sum_{k \geq 0} \frac{(-3)^{2 k}}{(2 k)!} z^{-2 k + 3}.
\end{align*}
Residyn i $0$ blir $\frac{81}{24}$ (svarar mot $k = 2$).
Därför blir integralen $\frac{27}{8} \pi i$.

\subsection{c)}
\begin{align*}
	\int_\gamma \frac{\dd{z}}{(z + 4) (z^2 + 1)}
\end{align*}

\paragraph{Lösning:}
Vi har två singulariteter som ligger i $\operatorname{int} \gamma$, $\pm i$.
Integranden går att skriva om som
\begin{align}
	h(z) = \frac{1}{(z + 4) (z + i) (z - i)}.
\end{align}
Kring $-i$ kan vi bryta ut $\frac{1}{z + i}$ och se att $h(z) \approx \frac{1}{(-i + 4) (-i - i} \frac{1}{z - i}$.
Så residyn i den punkten är $\frac{1}{(-i + 4) (-2i)}$ (enligt 9.6).
På samma sätt har vi att residyn i $i$ är $\frac{1}{(i + 4) (2i)}$.
Vi adderar dessa enligt residysatsen och erhåller att integralen blir $-\frac{2 \pi i}{17}$.

\subsection{d)}
\begin{align}
	\int_\gamma z^2 \Exp{\frac{1}{z}} \dd{z}.
\end{align}
\paragraph{Lösning:}
Vi Taylorutvecklar $\mathrm{exp}$ och får att
\begin{align*}
	\int_\gamma \sum_{k \geq 0} \frac{1}{k!} z^{-k + 2} \dd{z} ={}& 2 \pi i \residue_{z = 0} \left[\sum_{k \geq 0} \frac{1}{k!} z^{-k + 2}\right]\\
	={}& 2 \pi i \frac{1}{3!}\\
	={}& \frac{\pi i}{3}.
\end{align*}


\section{9.6}%
\label{sec:9_6}
Antag att en holo funktion $f$ har en pol av ordning $1$ i $a$ och att $g$ är holo i $a$.
Visa att
\begin{align}
	\residue_{z = a} [f(z) g(z)] = g(a) \residue_{z = a} [f(z)].
\end{align}

\paragraph{Lösning:}
Om $g$ är holo i $a$ kan vi Taylorutveckla henne där.
Vi kan också Laurentutveckla $f$ kring $a$.
Vi får då att
\begin{align}
	f(z) g(z) = \left( \sum_{k \geq -1} c_k (z - a)^k \right) \left( \sum_{k \geq 0} d_k (z - a)^k \right).
\end{align}
Från detta kan vi se att koefficienten för $\frac{1}{z}$ i $f g$:s Laurentutveckling är $c_{-1} d_0 = c_{-1} g(a)$.


\section{9.7}%
\label{sec:9_7}
Bestäm residyn i $0$ för följande funktioner.

\subsection{d)}
\begin{align*}
	\Exp{1 - \frac{1}{z}}.
\end{align*}

\paragraph{Lösning:}
Vi Taylorutvecklar $\mathrm{exp}$:
\begin{align*}
	\Exp{1 - \frac{1}{z}} = \sum_{k \geq 0} \frac{1}{k!} (1 - \frac{1}{z})^k.
\end{align*}
Härifrån kan vi se att $c_{-1} = \sum_{k \geq 1} \frac{-1}{k!} = -\exp{1} - 1$.
\todo[inline]{Facit sa $-\exp{}$ men det låter fel??}

\subsection{e)}
\begin{align*}
	\frac{\Exp{4 z} - 1}{\sin^2{z}}.
\end{align*}

\paragraph{Lösning:}
Jag vet inte om detta är det tydligaste sättet att göra det på, men vi skriver om uttrycket enligt
\begin{align*}
	\frac{\Exp{4 z} - 1}{\sin^2{z}} = \frac{\Exp{4 z} - 1}{z^2} \frac{z^2}{\sin^2{z}}.
\end{align*}
$\frac{z^2}{\sin^2{z}}$ är holo kring $0$.
Detta inses kanske lättast genom att använda standardgränsvärdet $\lim_{x \to 0} \frac{\sin{x}}{x} = 1$.
$\frac{\Exp{4 z} - 1}{z^2} = \frac{1}{z^2} \sum_{k \geq 1} \frac{1}{k!} (4 z)^k$ har en pol av ordning $1$ i $0$ med residyn $4$.


\section{9.8}%
\label{sec:9_8}
Använd residyer för att beräkna följande integraler.

\subsection{c)}
\begin{align*}
	\int_{D[0, 2]} \frac{\exp{z}}{z^3 + z} \dd{z}.
\end{align*}

\paragraph{Lösning:}
$z^3 + z = z (z - i) (z + i)$, så integranden har tre poler i $D[0, 2]$.
Vi erhåller
\begin{align*}
	\int_{D[0, 2]} \frac{\exp{z}}{z^3 + z} \dd{z}
	={}& 2 \pi i \sum_a \residue_{z = a} \left[ \frac{\exp{z}}{z (z - i) (z + i)} \right]\\
	%
	={}& 2 \pi i \left(
		\residue_{z = 0} \left[ \frac{\exp{z}}{z (z - i) (z + i)} \right]
		+ \residue_{z = i} \left[ \frac{\exp{z}}{z (z - i) (z + i)} \right]
		+ \residue_{z = -i} \left[ \frac{\exp{z}}{z (z - i) (z + i)} \right]
	\right)\\
	%
	={}& 2 \pi i \left(
		\frac{\exp{z}}{(z - i) (z + i)} \eval_{z = 0}
		+ \frac{\exp{z}}{z (z + i)} \eval_{z = i}
		+ \frac{\exp{z}}{z (z - i)}  \eval_{z = -i}
	\right)\\
	%
	={}& 2 \pi i \left(
		\frac{1}{-i^2} 
		+ \frac{\exp{i}}{2 i^2}
		+ \frac{\exp{-i}}{2 i^2}
	\right)\\
	%
	& \todo[inline]{TODO}
\end{align*}

\subsection{d)}
\paragraph{Lösning:}



