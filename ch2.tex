%! TEX root = /home/simon/Documents/komplexanalysrövningsledning/lösningsförslag/main.tex

\section{2.15}%
\label{sec:2_15}
Derivera $T(z) = \frac{a z + b}{c z + d}$ där $a$, $b$, $c$, och $d \in \complexes$ s.a.\ $a d - b c \neq 0$.
När är T'(z) = 0?

\paragraph{Lösning:}
\begin{align}
	T'(z) = \frac{a d - b c}{(c z + d)^2}.
\end{align}
Derivatan är därför aldrig $0$.


\section{2.21}%
\label{sec:2_21}
Bevisa att om $f(z)$ och $\conjugate{f(z)}$ båda är holomorfa på ett sammanhängande område $G \subset \complexes$, så är $f$ konstant på $G$.

\paragraph{Lösning:}
Från Cauchy--Riemann-ekvationen har vi att
\begin{align}
	\partialderivative{f}{x} = -i \partialderivative{f}{y},%
	\label{eq:CR_with_f}\\
	%
	\partialderivative{\conjugate f}{x} = -i \partialderivative{\conjugate f}{y},%
	\label{eq:CR_with_conjugate_f}
\end{align}
men om man komplexkonjugerar \cref{eq:CR_with_f} får man också att
\begin{align}
	\partialderivative{\conjugate f}{x} = i \partialderivative{\conjugate f}{y}.%
	\label{eq:conjugated_CR_with_f}
\end{align}
\cref{eq:CR_with_conjugate_f,eq:conjugated_CR_with_f} ger tillsammans att $f' = 0$ på $G$.

Vi kan då applicera Teorem 2.17 för att lösa uppgiften.


\section{2.22}%
\label{sec:2_22}
Antag att $f$ är hel och på formen $f(z) = u(x) + i v(y)$.
Visa att $f(z) = a z + b$ för något $a \in \reals$ och något $b \in \complexes$.

\paragraph{Lösning:}
Teorem 2.13 ger oss $f'(z) = u'(x)$, som är en reell funktion.
Vi kan då se att $f''(z)$ är väldefinerad och, eftersom $f''(z) = -i \partial_y f'(z)$, är $f''(z)$ $0$ överallt.
Enligt teorem 2.17 är då $f' = a$ för något konstant $a$ (som vi även vet är reell).
Eftersom $a z - f(z)$ har derivata $0$ kan vi applicera teorem 2.17 en gång till för att säga att $f(z) = a z + b$ för något $b \in \complexes$.


\section{2.23}%
\label{sec:2_23}
Antag att $f$ är hel med real- och imaginärdel $u$ och $v$ som uppfyller $u(x, y) v(x, y) = 3$ för alla på hela $\complexes$.
Visa att $f$ är konstant.

\paragraph{Lösning:}
\begin{align*}
	f(z)^2 ={}& \left(u(x, y) + i v(x, y)\right)^2\\
	={}& u(x, y)^2 + 2 i u(x, y) v(x, y) - v(x, y)^2\\
	={}& u(x, y)^2 + 6 i - v(x, y)^2.
\end{align*}
Alltså är $f(z)^2 - 6 i$ en reell hel funktion.
Som i uppgift 2.22 har vi då att $f(z)^2 - 6 i$ är en konstant funktion.


\section{2.25}%
\label{sec:2_25}
Hitta något $v$ s.a.\ $u + i v$ är holomorf i något område för följande $v$:

\subsection{a.}
$v(x, y) = x^2 - y^2.$

\paragraph{Lösning:}
$v = 2 x y$ ger $u + i v = z^2$.


\section{Linjär algebra och holomorfi}%
\label{sec:flervarre_tolkning_av_holomorfitet}

Om vi antar proposition 2.11 (bevaring av vinklar) som definitionen av holomorf kan vi härleda Cauchy--Riemanns ekvationer genom lite linjäralgebra.

En hyfsat rigorös definition av \emph{derivata} är följande.
\begin{definition}\label{def:derivata}
	Låt $E \subset \reals^n$ vara en öppen delmängd, och låt $f \from E \to \reals^m$ vara en funktion.
	$f$ är \emph{deriverbar} i en punkt $x \in E$ om det finns en linjär funktion $A \from \reals^n \to \reals^m$ s.a.\
	\begin{align}
		f(x + h) = f(x) + A h + \rho(h),
	\end{align}
	där $\frac{\rho(h)}{\abs{h}} \overset{h \to {\it 0}}{\longrightarrow} 0$.
\end{definition}

Om $E \subset \complexes$ så kan en funktion $f \from E \to \complexes$ även betraktas som en funktion $f \from \tilde E \to \reals^2$ där $\tilde E \subset \reals^2$ svarar mot $E$.
Från flervarren vet vi att om en funktion är partiellt deriverbar i ett öppet område är den deriverbar.
Så betrakta en sådan funktion $f$ och låt, utan inskränkning, $x = 0$.
Vi har då att
\begin{align}
	f(h) \approx f(0) + A h
\end{align}
för små $h$.
Vi har att $A$ är en $2 \times 2$-matris,
\begin{align}
	A =
	\begin{bmatrix}
		a_{11} & a_{12}\\
		a_{21} & a_{22}
	\end{bmatrix}.
\end{align}
Så $f$ är, nära $h$, en affin funktion som bestäms entydigt av var $0$ och basvektorerna avbildas.
Om $f$ ska vara holomorf måste vinkeln mellan bilden av basvektorerna vara $\frac{\pi}{2}$.
Dessa två villkor är ekvivalenta.

Därför måste $A$ vara proportionell mot en ortogonal matris.
Vi har att $A A^\mathrm{T} = \lambda I$ för något $\lambda \in \reals$.
Villkoren som matriselementen måste uppfylla är då
\begin{align}
	a_{12} ={}& -a_{21} \quad \text{och} \quad a_{11} = a_{22} \quad \text{eller}%
	\label{eq:CR_holo}\\
	%
	a_{12} ={}& a_{21} \quad \text{och} \quad a_{11} = -a_{22}.
	\label{eq:CR_antiholo}
\end{align}
Om vi nu identifierar $A$ med $f$:s Jacobianmatris
$\begin{bmatrix}
	\partial_x f_1 & \partial_y f_1\\
	\partial_x f_2 & \partial_y f_2
\end{bmatrix}$
ser vi att vi får tillbaka Cauchy--Riemanns ekvationer.
\cref{eq:CR_holo} beskriver villkoren för en holomorf funktion, medan \cref{eq:CR_antiholo} beskriver villkoren för en antiholomorf funktion.
